\chapter{Технологическая часть}

В данном разделе производится выбор средств реализации, а также приводятся требования к программному обеспечению (ПО), листинги реализованных алгоритмов.

\section{Требования к ПО}

На вход программе подаются две строки, а на выходе должно быть получено искомое расстояние, рассчитанное с помощью каждого реализованного алгоритма: для расстояния Левенштейна -- итеративный, для расстояния Дамерау-Левенштейна -- итеративный, рекурсивный без кэша, рекурсивный с кэшем. Также необходимо вывести затраченное каждым алгоритмом процессорное время.

Интерфейс создаваемого приложения должен предоставлять возможность ввода двух строк и выбора алгоритма. В качестве результата наглядно должен быть представлен график работы алгоритмов: зависимость времени от количества повторений операции, а также искомое расстояние.

\section{Выбор средств реализации}

В качестве языка программирования для реализации данной лабораторной работы был выбран язык Swift \cite{swift}. Данный язык создан компаний Apple для разработки программного обеспечения для macOS, iOS, watchOS и tvOS. Он содержит в себе большое количество инструментов, которые позволяют быстро создавать интерфейс приложений, а также реализовывать различные алгоритмы. Программное обеспечение, созданное посредством swift, позволит наглядно продемонстрировать скорость работы алгоритмов, а также упростить тестирование.

Кроме того, в Swift есть фреймворк CoreFoundation \cite{core}, который предоставляет функции для замера процессорного времени.

В качестве среды разработки выбран XCode \cite{xcode}. Альтернативой ему выступает среда AppCode от JetBrains \cite{appcode}. Однако Xcode, являясь официальной средой разработки Apple, предоставляет возможность запуска симуляторов устройств Apple, что играет ключевую роль при создании масштабируемых приложений.  

\section{Листинги кода}

В листингах 3.1 -- 3.4 представлены реализации рассматриваемых алгоритмов.

\section{Листинги кода}


\hspace{0.6cm}В Листинге 3.1 показана реализация матричного алгоритма нахождения расстояния Левенштейна.
\begin{lstlisting}[caption=Функция нахождения расстояния Левенштейна матрично]
func LevenshteinMatrix(_ firstString: String, _ secondString: String) -> Int {
    let n = firstString.count
    let m = secondString.count  
    var matrix = createMatrix(n: n + 1, m: m + 1, fill: 0)
    for i in 1...n {
        for j in 1...m {
            let symbolN = firstString.index(firstString.startIndex, offsetBy: i - 1)
            let symbolM = secondString.index(secondString.startIndex, offsetBy: j - 1)
            let insertOperation = matrix[i - 1][j] + 1
            let deleteOperation = matrix[i][j - 1] + 1
            var changeOperation = matrix[i - 1][j - 1]
            
            if firstString[symbolN] != secondString[symbolM] {
                changeOperation += 1
            }
            
            matrix[i][j] = min(insertOperation, deleteOperation, changeOperation)
        }
    }
    
    return matrix[n][m]
}
\end{lstlisting}


\hspace{0.6cm}В Листинге 3.2 показана реализация рекурсивного алгоритма нахождения расстояния Дамерау-Левенштейна.
\begin{lstlisting}[caption=Функция нахождения расстояния Дамерау-Левенштейна рекурсивно]
func DamerauLevenshteinRecursive(_ firstString: String, _ secondString: String) -> Int {
    let n = firstString.count
    let m = secondString.count
    if n == 0 || m == 0 {
        if n != 0 { return n }
        if m != 0 { return m }
        
        return 0
    }
    
    let symbolN = firstString.index(before: firstString.endIndex)
    let symbolM = secondString.index(before: secondString.endIndex)
    let a = String(firstString[..<symbolN])
    let b = String(secondString[..<symbolM])
    
    var change = 0
    if firstString[symbolN] != secondString[symbolM] {
        change += 1
    }
    
    let insertOperation = DamerauLevenshteinRecursive(a, secondString) + 1
    let deleteOperation = DamerauLevenshteinRecursive(firstString, b) + 1
    let changeOperation = DamerauLevenshteinRecursive(a, b) + change
    
    if n > 1 && m > 1 {
        let symbolNMinusOne = firstString.index(firstString.endIndex, offsetBy: -2)
        let symbolMMinusOne = secondString.index(secondString.endIndex, offsetBy: -2)
        let c = String(firstString[..<symbolNMinusOne])
        let d = String(secondString[..<symbolMMinusOne])
      
        if firstString[symbolN] == secondString[symbolMMinusOne] && firstString[symbolNMinusOne] == secondString[symbolM] {
            let transposeOperation = DamerauLevenshteinRecursive(c, d) + 1
            return min(insertOperation, deleteOperation, changeOperation, transposeOperation)
        }
    }
    
    return min(insertOperation, deleteOperation, changeOperation)
}
\end{lstlisting}

В Листинге 3.3 показана реализация матричного алгоритма нахождения расстояния Дамерау-Левенштейна.
\begin{lstlisting}[caption=Функция нахождения расстояния Дамерау-Левенштейна матрично]
func DamerauLevenshteinMatrix(_ firstString: String, _ secondString: String) -> Int {
    let n = firstString.count
    let m = secondString.count
    var matrix = createMatrix(n: n + 1, m: m + 1, fill: 0)
    
    for i in 1...n {
        for j in 1...m {
            let symbolN = firstString.index(firstString.startIndex, offsetBy: i - 1)
            let symbolM = secondString.index(secondString.startIndex, offsetBy: j - 1)
            
            let insertOperation = matrix[i - 1][j] + 1
            let deleteOperation = matrix[i][j - 1] + 1
            var changeOperation = matrix[i - 1][j - 1]
            if firstString[symbolN] != secondString[symbolM] {
                changeOperation += 1
            }
         
            if i > 1 && j > 1 {
                let symbolNMinusOne = firstString.index(firstString.startIndex, offsetBy: i - 2)
                let symbolMMinusOne = secondString.index(secondString.startIndex, offsetBy: j - 2)
                if firstString[symbolN] == secondString[symbolMMinusOne] && firstString[symbolNMinusOne] == secondString[symbolM] {
                    let transposeOperation = matrix[i - 2][j - 2] + 1
                    matrix[i][j] = min(insertOperation, deleteOperation, changeOperation, transposeOperation)
                } else {
                    matrix[i][j] = min(insertOperation, deleteOperation, changeOperation)
                }
            }
        }
    }
    return matrix[n][m]
}
\end{lstlisting}

В Листинге 3.4 показана реализация рекурсивного алгоритма нахождения расстояния Дамерау-Левенштейна с кэшем.
\begin{lstlisting}[caption=Функция нахождения расстояния Дамерау-Левенштейна рекурсивно с кэшем]
func DamerauLevenshteinRecursiveWithCash(_ firstString: String, _ secondString: String) -> Int {
    let n = firstString.count
    let m = secondString.count
    func recursive(firstString: String, secondString: String, n: Int, m: Int, matrix: inout [[Int]]) -> Int {
        if matrix[n][m] != -1 { return matrix[n][m] }
        let symbolN = firstString.index(firstString.startIndex, offsetBy: n - 1)
        let symbolM = secondString.index(secondString.startIndex, offsetBy: m - 1)
        var change = 0
        if firstString[symbolN] != secondString[symbolM] { change = 1 }
        let insertOperation = recursive(firstString, secondString, n - 1, m, &matrix) + 1
        let deleteOperation = recursive(firstString, secondString, n, m - 1, &matrix) + 1
        let changeOperation = recursive(firstString, secondString, n - 1, m - 1, &matrix) + change
        if n > 1 && m > 1 {
            let symbolNMinusOne = firstString.index(firstString.startIndex, offsetBy: n - 2)
            let symbolMMinusOne = secondString.index(secondString.startIndex, offsetBy: m - 2)
            matrix[n][m] = min(insertOperation, deleteOperation, changeOperation)
            if firstString[symbolN] == secondString[symbolMMinusOne] && firstString[symbolNMinusOne] == secondString[symbolM] {
                let transposeOperation = recursive(firstString, secondString, n - 2, m - 2, &matrix) + 1
                matrix[n][m] = min(insertOperation, deleteOperation, changeOperation, transposeOperation)
            }
        } else {
            matrix[n][m] = min(insertOperation, deleteOperation, changeOperation)
        }
        return matrix[n][m]
    }
    var matrix =  createMatrix(n: n + 1, m: m + 1, fill: -1)
    _ = recursive(firstString, secondString, n, m, &matrix)
    return matrix[n][m]
}
\end{lstlisting}

\section{Тестовые данные}

В таблице 3.1 приведены входные данные, на которых было протестированно разработанное ПО.

\begin{table}[h]
\renewcommand{\arraystretch}{1.8}
\renewcommand{\tabcolsep}{0.1cm} 
	\begin{center}
		
		\caption{Таблица тестовых данных}
		""\newline
		\begin{tabular}{|c c c c c| } 
			\hline
			№ & Первое слово & Второе слово & Ожидаемый результат & Полученный результат \\ [0.8ex] 
			\hline
			1 & а & б & 1 1 1 1 & 1 1 1 1\\
			\hline
			2 & кот & скат & 2 2 2 2 & 2 2 2 2\\
			\hline
			3 & море & моер & 2 1 1 1 & 2 1 1 1\\
			\hline
			4 & абаваба & аабвааб & 4 2 2 2 & 4 2 2 2\\
			\hline
			5 & собака & собака & 0 0 0 0 & 0 0 0 0\\
			\hline
			6 & qwerty & queue & 4 4 4 4 & 4 4 4 4\\
			\hline
			7 & apple & aplpe & 2 1 1 1  & 2 1 1 1\\
			\hline
			8 & 12 & one & 3 3 3 3 & 3 3 3 3\\
			\hline
			9 & календула & конфета & 6 6 6 6 & 6 6 6 6\\
			\hline
			10 & конфета & календула & 6 6 6 6 & 6 6 6 6\\
			\hline
		\end{tabular}
	\end{center}
\end{table}


\section*{Вывод}

Был производен выбор средств реализации и реализованы алгоритмы поиска расстояний: Левештнейна -- итеративный, Дамерау-Левенштейна -- итеративный, рекурсивный с кэшем и рекурсивный без кэша.
Приведены листинги кода на выбранном языке программирования, а также представлена таблица, отображающая результаты работы программы на предложенных наборах тестовых данных.