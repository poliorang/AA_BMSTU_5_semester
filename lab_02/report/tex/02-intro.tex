\chapter*{Введение}
\addcontentsline{toc}{chapter}{Введение}

Матрица — математический объект, записываемый в виде прямоугольной таблицы элементов, который представляет собой совокупность строк и столбцов, на пересечении которых находятся его элементы. Матрицы встречаются не только в деятельности науки, но и в повседневной жизни, ведь любое представление информации в виде таблицы тоже является матрицей. 

Целью работы является изучение и реализация алгоритмов
умножения матриц, вычисление трудоёмкости этих алгоритмов. В данной
лабораторной работе рассматривается стандартный алгоритм умножения
матриц, алгоритм Винограда и модифицированный алгоритм Винограда.

Для достижения цели ставятся следующие задачи.

\begin{enumerate}
	\item Изучить классический алгоритм умножения матриц, алгоритм Винограда и модифицированный алгоритм Винограда.
	\item Реализовать классический алгоритм умножения матриц, алгоритм
	Винограда и модифицированный алгоритм Винограда.

	\item Дать оценку трудоёмкости алгоритмов.
	\item Замерить время работы алгоритмов.
	\item Провести сравнительный анализ на основе полученных экспериментально данных.
\end{enumerate}
