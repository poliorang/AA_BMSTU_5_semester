\chapter{Аналитическая часть}

\section{Стандартный алгоритм}

К операциям, которые можно совершать над матрицами, относятся сложение, вычисание, умножение на число, возведение в степень и т.д. В один ряд с ними стоит операция умножения матриц. 


Пусть даны две прямоугольные матрицы
\begin{equation}
	A_{lm} = \begin{pmatrix}
		a_{11} & a_{12} & \ldots & a_{1m}\\
		a_{21} & a_{22} & \ldots & a_{2m}\\
		\vdots & \vdots & \ddots & \vdots\\
		a_{l1} & a_{l2} & \ldots & a_{lm}
	\end{pmatrix},
	\quad
		B_{mn} = \begin{pmatrix}
		b_{11} & b_{12} & \ldots & b_{1n}\\
		b_{21} & b_{22} & \ldots & b_{2n}\\
		\vdots & \vdots & \ddots & \vdots\\
		b_{m1} & b_{m2} & \ldots & b_{mn}
	\end{pmatrix},
\end{equation}

тогда матрица $C$
\begin{equation}
	C_{ln} = \begin{pmatrix}
		c_{11} & c_{12} & \ldots & c_{1n}\\
		c_{21} & c_{22} & \ldots & c_{2n}\\
		\vdots & \vdots & \ddots & \vdots\\
		c_{l1} & c_{l2} & \ldots & c_{ln}
	\end{pmatrix},
\end{equation}

где
\begin{equation}
	\label{eq:M}
	c_{ij} =
		\sum_{r=1}^{m} a_{ir}b_{rj} \quad (i=\overline{1,l}; j=\overline{1,n})
\end{equation}

будет называться произведением матриц $A$ и $B$.
Стандартный алгоритм реализует формулу  \ref{eq:M}.

\section{Алгоритм Винограда}

Можно заметить, что каждый элемент матрицы, полученной при умножении двух других, представляет собой скалярное произведение соответствующих строки и столбца исходных матриц. А также -- дабы выполнить часть работы заранее --  такое умножение допускает предварительную обработку.

Рассмотрим два вектора $V = (v_1, v_2, v_3, v_4)$ и $W = (w_1, w_2, w_3, w_4)$.

Их скалярное произведение равно: $V \cdot W = v_1w_1 + v_2w_2 + v_3w_3 + v_4w_4$, что эквивалентно (\ref{for:new}):
\begin{equation}
	\label{for:new}
		V \cdot W = (v_1 + w_2) (v_2 + w_1) + (v_3 + w_4)(v_4 + w_3) - v_1v_2 - v_3v_4 - w_1w_2 - w_3w_4.
\end{equation}

Для примера, приведенного в формуле \ref{for:new}, в стандартном алгоритме производятся четыре умножения и три сложения, в алгоритме Винограда -- шесть умножений и десять сложений \cite{vin}. Но, несмотря на увеличение количества операций, выражение в правой части последнего равенства допускает предварительную обработку: его части можно вычислить заранее и запомнить для каждой строки первой матрицы и для каждого столбца второй, что позволит для каждого элемента выполнять лишь два умножения и пять сложений, складывая затем только лишь с двумя предварительно вычисленными суммами соседних элементов текущих строк и столбцов. Как правило, в ЭВМ операция сложения быстрее операции умножения, поэтому алгоритм Винограда должен работать быстрее стандартного.

\section{Вывод}
	В данном разделе были рассмотрены алгоритмы классического умножения матриц и алгоритм Винограда, основное отличие которого от классического алгоритма — наличие предварительной обработки, а также количество операций умножения и сложения. 
\clearpage