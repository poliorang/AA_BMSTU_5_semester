\chapter{Технологическая часть}

В данном разделе производится выбор средств реализации, а также приводятся требования к программному обеспечению (ПО), листинги реализованных алгоритмов.

\section{Требования к ПО}

На вход программе подаются две матрицы, на выходе должна быть получена результирующая матрица - произведение двух первых, вычисленная с помощью каждого реализованного алгоритма: стандартного, Винограда, Винограда с оптимизацией. Также необходимо вывести затраченное каждым алгоритмом процессорное время.

Интерфейс создаваемого приложения должен предоставлять возможность ввода двух матриц, задания размерностей и генерации по ним двух матриц и выбора алгоритма умножения. В качестве результата должна быть выведена результирующая матрица и наглядно представлен график работы алгоритмов: зависимость времени от количества повторений операции.

\section{Выбор средств реализации}

В качестве языка программирования для реализации данной лабораторной работы был выбран язык Swift \cite{swift}. Данный язык создан компаний Apple для разработки программного обеспечения для macOS, iOS, watchOS и tvOS. Он содержит в себе большое количество инструментов, которые позволяют быстро создавать интерфейс приложений, а также реализовывать различные алгоритмы. Программное обеспечение, созданное посредством swift, позволит наглядно продемонстрировать скорость работы алгоритмов, а также упростить тестирование.

Кроме того, в Swift есть фреймворк CoreFoundation \cite{core}, который предоставляет функции для замера процессорного времени.

В качестве среды разработки выбран XCode \cite{xcode}. Альтернативой ему выступает среда AppCode от JetBrains \cite{appcode}. Однако Xcode, являясь официальной средой разработки Apple, предоставляет возможность запуска симуляторов устройств Apple, что играет ключевую роль при создании масштабируемых приложений.  

\section{Листинги кода}

В листингах 3.1 -- 3.3 представлены реализации рассматриваемых алгоритмов.

\section{Листинги кода}


\hspace{0.6cm}В Листинге 3.1 показана реализация стандартного алгоритма умножения матриц.
\begin{lstlisting}[caption=Функция стандартного умножения матриц]
func standardMultiplication(_ matrixA: [[Int]], _ matrixB: [[Int]]) -> [[Int]] {

    let n = matrixA.count
    let m = matrixA[0].count
    let q = matrixB[0].count
    
    resultMatrix = createMatrix(n: n, m: q, fill: 0)
    
    for i in 0..<n {
        for j in 0..<q {
            for k in 0..<m {
                resultMatrix[i][j] =  \
                resultMatrix[i][j] + matrixA[i][k] * matrixB[k][j]
            }
        }
    }
    
    return resultMatrix
}
\end{lstlisting}





\hspace{0.6cm}В Листинге 3.2 показана реализация реализация алгоритма Винограда умножения матриц.
\begin{lstlisting}[caption=Функция алгоритма Винограда умножения матриц]
func Winograd(_ matrixA: [[Int]], _ matrixB: [[Int]]) -> [[Int]] {
    let n = matrixA.count
    let m = matrixA[0].count
    let q = matrixB[0].count
    
    resultMatrix = [[Int]](repeating: [Int](repeating: 0, count: q), count: n)
    var row = [Int](repeating: 0, count: n)
    
    for i in 0..<n {
        for j in 0..<(m / 2) {
            row[i] = row[i] + matrixA[i][2 * j] * matrixA[i][2 * j + 1]
        }
    }
    
    var column = [Int](repeating: 0, count: q)
    for i in 0..<q {
        for j in 0..<(m / 2) {
            column[i] = column[i] + matrixB[2 * j][i] * matrixB[2 * j + 1][i]
        }
    }
    
    for i in 0..<n {
        for j in 0..<q {
            resultMatrix[i][j] = -row[i] - column[j]
            for k in 0..<(m / 2) {
                resultMatrix[i][j] = resultMatrix[i][j] + (matrixA[i][2 * k + 1] + matrixB[2 * k][j]) * (matrixA[i][2 * k] + matrixB[2 * k + 1][j])
            }
        }
    }
    
    if m % 2 == 1 {
        for i in 0..<n {
            for j in 0..<q {
                resultMatrix[i][j] = resultMatrix[i][j] + matrixA[i][m - 1] * matrixB[m - 1][j]
            }
        }
    }
    
    return resultMatrix
}
\end{lstlisting}

В Листинге 3.3 показана реализация оптимизированного алгоритма Винограда умножения матриц.
\begin{lstlisting}[caption=Функция оптимизированного алгоритма Винограда умножения матриц]
func WinogradOptimal(_ matrixA: [[Int]], _ matrixB: [[Int]]) -> [[Int]] {
    let n = matrixA.count
    let m = matrixA[0].count
    let mCycle = matrixA[0].count >> 1
    let q = matrixB[0].count
    
    resultMatrix = [[Int]](repeating: [Int](repeating: 0, count: q), count: n)
    var row = [Int](repeating: 0, count: n)
    
    
    for i in 0..<n {
        for j in 0..<mCycle {
            row[i] += matrixA[i][2 * j] * matrixA[i][2 * j + 1]
        }
    }
    
    var column = [Int](repeating: 0, count: q)
    
    for i in 0..<q {
        for j in 0..<mCycle {
            column[i] += matrixB[2 * j][i] * matrixB[2 * j + 1][i]
        }
    }
    for i in 0..<n {
        for j in 0..<q {
            resultMatrix[i][j] -= (row[i] + column[j])
            for k in 0..<mCycle {
                resultMatrix[i][j] += (matrixA[i][2 * k + 1] + matrixB[2 * k][j]) * (matrixA[i][2 * k] + matrixB[2 * k + 1][j])
            }
        }
    }
    if m % 2 == 1 {
        for i in 0..<n {
            for j in 0..<q {
                resultMatrix[i][j] += matrixA[i][m - 1] * matrixB[m - 1][j]
            }
        }
    }
    return resultMatrix
}
\end{lstlisting}


\section{Функциональные тесты}

В таблице \ref{tab:tests} приведены тесты для функций, реализующих алгоритмы умножения матриц, рассматриваемых в данной лабораторной работе. Тесты для всех алгоритмов пройдены успешно.


\begin{table}[h]
	\begin{center}
		\caption{\label{tab:tests} Функциональные тесты}
		\begin{tabular}{|c|c|c|c|c|}
		\hline
		Матрица 1 & Матрица 2 & Ожидаемый результат \\ 
		\hline

		$\begin{pmatrix}
			&
		\end{pmatrix}$ &
		$\begin{pmatrix}
			&
		\end{pmatrix}$ &
		Сообщение об ошибке \\ \hline
			
		$\begin{pmatrix}
			1 & 5 & 7\\
			2 & 6 & 8\\
			3 & 7 & 9
		\end{pmatrix}$ &
		$\begin{pmatrix}
			&
		\end{pmatrix}$ &
		Сообщение об ошибке \\ \hline

		$\begin{pmatrix}
			1 & 2 & 3
		\end{pmatrix}$ &
		$\begin{pmatrix}
			1 & 2 & 3
		\end{pmatrix}$ &
		Сообщение об ошибке \\ \hline

		$\begin{pmatrix}
			1 \\
			1 \\
			1
		\end{pmatrix}$ &
		$\begin{pmatrix}
			1 \\
			1 \\
			1
		\end{pmatrix}$ &
		Сообщение об ошибке \\ \hline

		$\begin{pmatrix}
			1 & a & 3
		\end{pmatrix}$ &
		$\begin{pmatrix}
			1 \\
			1 \\
			1
		\end{pmatrix}$ &
		Сообщение об ошибке \\ \hline
			
		$\begin{pmatrix}
			1 & 2 & 3
		\end{pmatrix}$ &
		$\begin{pmatrix}
			1 \\
			a \\
			1
		\end{pmatrix}$ &
		Сообщение об ошибке \\ \hline

		$\begin{pmatrix}
			1 & 1 & 1
		\end{pmatrix}$ &
		$\begin{pmatrix}
			1 \\
			1 \\
			1
		\end{pmatrix}$ &
		$\begin{pmatrix}
			1 & 1 & 1\\
			1 & 1 & 1 \\
			1 & 1 & 1
		\end{pmatrix}$ \\ \hline

		$\begin{pmatrix}
			1 & 2 & 3 \\
			4 & 5 & 6 \\
			7 & 8 & 9
		\end{pmatrix}$ &
		$\begin{pmatrix}
			1 & 0 & 0 \\
			0 & 1 & 0 \\
			0 & 0 & 1
		\end{pmatrix}$ &
		$\begin{pmatrix}
			1 & 2 & 3 \\
			4 & 5 & 6 \\
			7 & 8 & 9
		\end{pmatrix}$ \\ \hline

		$\begin{pmatrix}
			1 & 2 & 3 \\
			4 & 5 & 6 \\
			7 & 8 & 9
		\end{pmatrix}$ &
		$\begin{pmatrix}
			1 & 1 \\
			1 & 1 \\
			1 & 1 
		\end{pmatrix}$ &
		$\begin{pmatrix}
			6 & 6 \\
			15 & 15 \\
			24 & 24
		\end{pmatrix}$ \\ \hline

		$\begin{pmatrix}
			2
		\end{pmatrix}$ &
		$\begin{pmatrix}
			2
		\end{pmatrix}$ &
		$\begin{pmatrix}
			4
		\end{pmatrix}$ \\ \hline

		\end{tabular}
	\end{center}
\end{table}


\section*{Вывод}

В данном разделе были разработаны исходные коды четырёх алгоритмов перемножения матриц: обычный алгоритм, алгоритм Винограда, Винограда с оптимизацией.
