\chapter*{Введение}
\addcontentsline{toc}{chapter}{Введение}


Сортировкой называют процесс перегруппировки заданной последовательности объектов в некотором определенном порядке. Определенный порядок (например, упорядочение последовательности целых чисел по возрастанию) в последовательности объектов необходим для удобства работы с этим объектом. Одной из целей сортировки является упрощение дальнейшего поиска элементов в отсортированном множестве. Существует множество различных методов сортировки данных. Однако любой алгоритм сортировки можно разбить на три основные части:

\begin{itemize}
	\item сравнение, определяющее упорядоченность пары элементов;
	\item перестановка, меняющая местами пару элементов;
	\item сортирующий алгоритм, который осуществляет сравнение и перестановку элементов данных до тех пор, пока все эти элементы не будут упорядочены.
\end{itemize}

Целью работы работы является изучение и реализация сортировки, вычисление трудоёмкости этих алгоритмов. В данной
лабораторной работе рассматривается блочная сортировка, сортировка перемешиванием и бинарным деревом.

Для достижения цели ставятся следующие задачи:
\begin{itemize}
	\item изучить и реализовать 3 алгоритма сортировки: блочная, перемешиванием и бинарным деревом;
	\item провести сравнительный анализ трудоёмкости алгоритмов на основе теоретических расчетов и выбранной модели вычислений;
	\item провести сравнительный анализ времени работы алгоритмов на основе экспериментальных данных и памяти, используемой ими. 
\end{itemize}
