\chapter{Аналитическая часть}

\section{Блочная сортировка}

Блочная сортировка \cite{bucket} -- это алгоритм сортировки, который разделяет несортированные элементы массива на несколько групп, называемых блоками или корзинами. Затем каждая корзина сортируется с использованием любого из подходящих алгоритмов сортировки или рекурсивного применения того же алгоритма блочной сортировки: мной был выбран алгоритм сортировки вставками в силу быстроты работы с почти упорядоченными массивами малых размеров. 

Мной были найдены реализации данной сортировки только для положительных чисел, поэтому этап поиска интервала -- размера блока был усовершенствован, дабы алгоритм был применим и для отрицательных чисел. 

В итоге отсортированные сегменты объединяются для формирования окончательного отсортированного массива.

\section{Сортировка перемешиванием}

Сортировка перемешиванием \cite{shaker} -- это разновидность сортировки пузырьком. Отличие в том, что данная сортировка в рамках одной итерации проходит по массиву в обоих направлениях (слева направо и справа налево), тогда как сортировка пузырьком -- только в одном направлении (слева направо).

Общие идеи алгоритма:
\begin{itemize}
	\item обход массива слева направо, аналогично пузырьковой -- сравнение соседних элементов, меняя их местами, если левое значение больше правого;
	\item обход массива в обратном направлении (справа налево), начиная с элемента, который находится перед последним отсортированным, то есть на этом этапе элементы также сравниваются между собой и меняются местами, чтобы наименьшее значение всегда было слева.
\end{itemize}

\section{Сортировка бинарным деревом}

Из элементов массива формируется бинарное дерево поиска \cite{tree}. Первый элемент - корень дерева, остальные добавляются по следующему методу: начиная с корня дерева, элемент сравнивается с узлами. Если элемент меньше чем узел -- спускаемся по левой ветке, иначе -- по правой. Спустившись до конца, сам элемент становится узлом.

Построенное таким образом дерево можно легко обойти так, чтобы двигаться от узлов с меньшими значениями к узлам с большими значениями. При этом получаем все элементы в возрастающем порядке.

\section{Вывод}

В данном разделе были рассмотрены и описаны три алгоритма сортировок: блочная, перемешиванием, бинарным деревом. 
